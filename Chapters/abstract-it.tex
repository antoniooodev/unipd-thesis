%!TEX root = ../template.tex
%%%%%%%%%%%%%%%%%%%%%%%%%%%%%%%%%%%%%%%%%%%%%%%%%%%%%%%%%%%%%%%%%%%%
%% abstract-it.tex
%% UNIPD thesis document file
%%
%% Abstract in Italian
%%%%%%%%%%%%%%%%%%%%%%%%%%%%%%%%%%%%%%%%%%%%%%%%%%%%%%%%%%%%%%%%%%%%

\typeout{NT FILE abstract-it.tex}%


Questa tesi esplora lo sviluppo di un innovativo sistema di streaming audio per ambienti ristorativi, basato su tecnologia Raspberry Pi. Il lavoro si concentra sull'integrazione di sistemi embedded e Internet of Things (IoT) per creare microambienti sonori personalizzati, mirando a migliorare l'esperienza gastronomica attraverso il controllo audio individuale. L'indagine si estende all'applicazione di reti neurali , ottimizzate con l'algoritmo Harris Hawks, per analizzare la relazione tra rumore ambientale e gradimento del cibo. Questo studio apre nuove prospettive nell'interazione tra tecnologia audio, percezione sensoriale e design degli spazi ristorativi.


% Palavras-chave do resumo em Italiano
\keywords{
  Raspberry Pi \and
  Streaming audio \and
  Internet of Things (IoT) \and
  Reti neurali \and
  Harris Hawks \and
}
