%!TEX root = ../template.tex
%%%%%%%%%%%%%%%%%%%%%%%%%%%%%%%%%%%%%%%%%%%%%%%%%%%%%%%%%%%%%%%%%%%%
%% abstract-it.tex
%% UNIPD thesis document file
%%
%% Abstract in Italian
%%%%%%%%%%%%%%%%%%%%%%%%%%%%%%%%%%%%%%%%%%%%%%%%%%%%%%%%%%%%%%%%%%%%

\typeout{NT FILE abstract-it.tex}%


Questa tesi presenta l'architettura di un sistema di streaming audio basato su Raspberry Pi per il controllo di microambienti sonori, applicato al settore della ristorazione. Il progetto integra tecnologie embedded e IoT per creare microambienti sonori personalizzati, migliorando l'esperienza culinaria attraverso il controllo audio individuale. Vengono esplorate soluzioni avanzate per l'ottimizzazione dell'acustica nei ristoranti, con l'obiettivo di migliorare l'interazione sensoriale e il comfort degli utenti.

\keywords{
  Raspberry Pi \and
  Streaming audio system \and
  Internet of Things (IoT) \and
  Reti neurali \and
  Harris Hawks \and
}
