%!TEX root = ../template.tex
%%%%%%%%%%%%%%%%%%%%%%%%%%%%%%%%%%%%%%%%%%%%%%%%%%%%%%%%%%%%%%%%%%%
%% chapter6.tex
%% UNIPD thesis document file
%%
%% Chapter with introduction
%%%%%%%%%%%%%%%%%%%%%%%%%%%%%%%%%%%%%%%%%%%%%%%%%%%%%%%%%%%%%%%%%%%

\typeout{NT FILE chapter6.tex}%

\chapter{Conclusioni, Limitazioni e Proposte di Sviluppo Futuro}
\label{cha:conclusioni}

La scelta del Raspberry Pi come hardware ha offerto vantaggi in termini economici e di flessibilità; tuttavia, la sua potenza  potrebbe risultare insufficiente per applicazioni audio più complesse o per la gestione di ambienti di grandi dimensioni. Ad esempio, dispositivi come l'ASUS Tinker Board 2S offrono prestazioni superiori che possono migliorare la qualità dello streaming e la gestione simultanea di più flussi. Allo stesso modo, la NVIDIA Jetson Nano, è progettata per applicazioni che richiedono elevate capacità di calcolo, come l'elaborazione audio avanzata con l'integrazione dell'intelligenza artificiale.

Inoltre, la maggior parte dei test è stata condotta in ambienti controllati. È noto che le condizioni reali possono introdurre variabili impreviste, come interferenze elettromagnetiche o ostacoli fisici, che possono intaccare nella qualità dell'audio. 

Il modello ANN-HHO ha dato risultati promettenti ma si tratta di uno studio ormai superato, ma pur sempre fondamentale. L'integrazione con modelli di deep learning, come le reti neurali \gls{cnn} o le reti \gls{rnn}, potrebbe offrire prestazioni superiori nella gestione e nell'analisi dei dati. \cite{8727654, 9717998} \cite{8325661} \cite{9680690, 9840922}

L'implementazione di algoritmi di ottimizzazione più recenti, come l'ottimizzazione basata su sciami di particelle (\gls{pso}) o l'algoritmo genetico (\gls{ga}), potrebbe migliorare la capacità del sistema di adattarsi a variabili ambientali piu' complesse. Queste tecniche sono state oggetto di numerosi studi recenti che ne evidenziano l'efficacia in applicazioni simili. L'uso di \gls{cnn}, ad esempio, per l'elaborazione del segnale audio può migliorare significativamente la qualità dello streaming in ambienti con elevato rumore di fondo. O come l'applicazione di \gls{pso} in sistemi di ottimizzazione ha mostrato una maggiore efficienza nel raggiungimento di soluzioni ottimali rispetto a metodi tradizionali. \cite{10629358, 10258355}

\newpage
\hspace{2cm}
\begin{quote}
L'adozione di queste tecniche rappresenterebbe un passo decisivo verso un sistema capace di fondere l'esperienza sensoriale nella quotidianità della ristorazione, trasformando ogni ambiente in un luogo in cui la tecnologia e la percezione umana si incontrano armoniosamente. 
\end{quote}

\centering
\t{oo}