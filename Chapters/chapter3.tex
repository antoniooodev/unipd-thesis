%!TEX root = ../template.tex
%%%%%%%%%%%%%%%%%%%%%%%%%%%%%%%%%%%%%%%%%%%%%%%%%%%%%%%%%%%%%%%%%%%
%% chapter3.tex
%% UNIPD thesis document file
%%
%% Chapter with introduction
%%%%%%%%%%%%%%%%%%%%%%%%%%%%%%%%%%%%%%%%%%%%%%%%%%%%%%%%%%%%%%%%%%%

\typeout{NT FILE chapter3.tex}%

\chapter{Fondamenti Teorici}
\label{cha:introduzione}

\prependtographicspath{{Chapters/Figures/Covers/}}

\section{Sistemi embedded e IoT}

\subsection{Introduzione ai sistemi embedded}
\noindent

I sistemi embedded sono sistemi di elaborazione integrati all'interno dell'oggetto o del sistema informatico, progettati per compiti specifici, che offrono un'elevata efficienza e un basso consumo energetico rispetto ai sistemi general-purpose. Sono sempre più utilizzati in applicazioni come l'elaborazione audio e i sistemi di controllo in tempo reale in vari ambienti, tra cui l'industria della ristorazione. Grazie alla loro flessibilità, questi sistemi possono integrarsi perfettamente in ambienti complessi dove l'affidabilità e la compattezza sono fattori critici.

Nel contesto dei sistemi audio per ristoranti, le piattaforme embedded come Raspberry Pi sono ideali per gestire flussi audio in tempo reale, decodificare l'audio digitale e garantire uscite audio sincronizzate per le diverse zone di un ristorante, come i singoli tavoli.

La loro scalabilità ed economicità le rendono adatte a implementare ambienti audio personalizzati in locali di medie e grandi dimensioni.


\subsection{L'\gls{iot} in ambito ristorativo}
\noindent

L'\gls{iot} ha rivoluzionato il settore della ristorazione, consentendo ai dispositivi di interagire e scambiare dati in modo autonomo. Nei ristoranti intelligenti, i dispositivi \gls{iot} possono gestire tutto, dal servizio clienti al controllo dell'ambiente. Una delle applicazioni più innovative è l'uso dell'\gls{iot} per creare ambienti sonori personalizzati, o “bolle sonore”, dove ogni tavolo può avere il proprio percorso sonoro su misura.

L'integrazione di sensori e dispositivi audio consente ai ristoranti di monitorare e regolare in tempo reale i livelli audio, il rumore ambientale e le preferenze dei clienti. Ad esempio, la musica di sottofondo può essere regolata automaticamente in base ai dati in tempo reale, come il feedback dei clienti o i livelli di rumore.

Le tecnologie \gls{iot} in questo contesto offrono un'esperienza utente più fluida , aumentando la soddisfazione dei clienti e l'efficienza operativa.

L'implementazione di queste tecnologie non si limita a migliorare l'esperienza audio nei ristoranti ma contribuisce anche a migliorare la sicurezza. I sensori  possono essere applicati per monitorare l'ambiente della cucina e inviare avvisi in caso di pericolo di incendio o possono monitorare le operazioni di cucina, rendendo il ristorante più reattivo ed efficiente.

\subsection{Raspberry Pi come piattaforma per sistemi audio}
\noindent

Il Raspberry Pi è ampiamente riconosciuto come piattaforma ideale per i sistemi embedded nelle applicazioni audio, grazie al suo basso costo, alla versatilità e alla sufficiente potenza di elaborazione.

In questo progetto di tesi, il Raspberry Pi Zero 2W viene utilizzato come unità di elaborazione centrale per gestire i flussi audio e distribuirli ai singoli dispositivi sul tavolo. Il sistema sfrutta Mopidy, un server di streaming basato su Python, insieme a Iris per i controlli frontend, consentendo la gestione in tempo reale di playlist e livelli audio. L'integrazione del Pi con \gls{dac} e \gls{gpio} consente inoltre di formare un'architettura di sistema flessibile che può essere ampliata con sensori o interfacce audio aggiuntive.

Il supporto di Raspberry Pi per il multitasking e la sua capacità di gestire l'elaborazione in parallelo lo rendono uno dei migliori candidati per le attività di elaborazione audio in un ristorante, dove è necessario gestire in modo efficiente più canali.

Inoltre, la compatibilità di questo dispositivo con varie librerie software,  consente al sistema di essere altamente personalizzabile e adattabile a diversi ambienti.

\section{Tecnologie di streaming audio}

\subsection{Panoramica delle tecnologie di streaming audio}
\noindent

Lo streaming audio si riferisce alla trasmissione in tempo reale di dati audio su una rete, consentendo agli utenti di ascoltare l'audio senza doverlo scaricare in locale. Questa tecnologia si basa su protocolli che gestiscono il flusso costante di pacchetti di dati tra server e client  per garantire una riproduzione fluida. Due dei protocolli più utilizzati sono il \gls{rtsp} e l'\gls{hls}.

\begin{itemize}
    \item \gls{rtsp} stabilisce e controlla uno o più flussi sincronizzati nel 
          tempo di media continui, come audio. In genere non fornisce 
          direttamente i flussi continui, anche se è possibile l'interleaving del 
          flusso multimediale continuo con il flusso di controllo. In altre parole, 
          \gls{rtsp} agisce come un ``telecomando di rete'' per i server 
          multimediali.
          
    \item \gls{hls}, è un formato adattativo basato su \gls{http} per il trasporto 
          di dati video e audio dai server multimediali agli schermi degli 
          spettatori. I contenuti video e audio vengono suddivisi in una serie di 
          pezzi, compressi per una consegna rapida e trasmessi ai dispositivi 
          degli utenti finali. Gli spettatori godono quindi di uno streaming fluido, 
          nonostante tutto ciò che accade in background. La popolarità di 
          questo protocollo rispetto alle sue alternative è dovuta alla 
          compatibilità della riproduzione e alla qualità dell'esperienza. Infatti, 
          tutti i dispositivi Mac, Android, Microsoft e Linux sono in grado di 
          riprodurre flussi trasmessi con \gls{hls}.
  \end{itemize}

Entrambi i protocolli sono essenziali per mantenere una trasmissione audio a bassa latenza e di alta qualità, particolarmente importante in scenari che richiedono la sincronizzazione di più flussi in zone diverse.

\subsection{Mopidy: Architettura e caratteristiche}
\noindent

Mopidy è un servizio open-source  basato su Python, progettato per gestire e riprodurre audio sia in locale che online. Funziona come un server \gls{http}, consentendo agli utenti di controllare la musica attraverso interfacce web e altri dispositivi remoti. Mopidy viene spesso utilizzato in ambienti in cui è richiesto lo streaming audio multi-room, offrendo una soluzione flessibile e personalizzabile per spazi pubblici e privati. Questo sistema supporta diverse interfacce frontend, come Iris, e può integrarsi con piattaforme come Spotify, SoundCloud, e file system locali.

Uno dei punti di forza di Mopidy è la sua architettura modulare, che consente agli sviluppatori di estendere le sue capacità attraverso un'ampia gamma di plugin. Questa modularità rende Mopidy altamente adattabile a diverse applicazioni, dalle configurazioni audio domestiche agli ambienti commerciali in cui l'audio deve essere trasmesso in più zone.

L'uso dell'architettura client-server di Mopidy è particolarmente adatto alla gestione di più flussi audio in rete. Ogni client può essere collegato a una serie di altoparlanti, mentre il server gestisce la riproduzione del file audio. Mopidy spesso è abbinato a Snapcast, un altro strumento che sincronizza l'audio tra più client. Snapcast assicura che tutti i client rimangano sincronizzati, riproducendo lo stesso flusso audio senza ritardi evidenti. Questo sistema consente un controllo preciso sulla riproduzione audio in più luoghi, rendendolo ideale per spazi pubblici come scuole, uffici o campus di più edifici.

Il server Mopidy può essere controllato da remoto utilizzando una varietà di interfacce, consentendo agli utenti di gestire playlist, controllare il volume e programmare la riproduzione utilizzando interfacce basate sul web e a riga di comando. Questa flessibilità rende Mopidy una soluzione molto efficace per gli ambienti in cui l'audio deve essere gestito da una postazione centrale.~\cite{cit-mopidy}

\section{Acustica nei ristoranti}

\subsection{L'importanza dell'ambiente acustico nei ristoranti}
\noindent

Uno degli elementi chiave che definiscono la qualità acustica di uno spazio è il \gls{rt60}, che misura il tempo necessario del suono per dissolversi in un determinato ambiente. Nei ristoranti, ad esempio, un riverbero elevato può compromettere la chiarezza di un discorso, costringendo i clienti ad alzare la voce, con conseguente aumento dei livelli di rumore. Di conseguenza, un tempo di riverberazione ben controllato consente una comunicazione più chiara, creando un ambiente più confortevole per il cliente.

Allo stesso modo, l'equilibrio tra assorbimento e riflessione del suono gioca un ruolo cruciale nel determinare il comfort acustico complessivo. Materiali come pannelli acustici, tappeti e imbottiture possono essere posizionati strategicamente per ridurre la riflessione delle onde sonore. Quando le superfici sono molto riflettenti, creano echi e sovrapposizioni di onde sonore, che possono rendere l'ambiente caotico e sgradevole.

L'acustica e il comportamento umano sono correlati. Alti livelli di rumore nei ristoranti sono spesso associati a un aumento dello stress, a una riduzione del comfort e persino a una percezione negativa del cibo e del servizio. Alcuni studi hanno dimostrato che gli ambienti rumorosi possono portare a soggiorni più brevi e a un consumo più rapido dei pasti, in quanto i clienti possono trovare difficile rilassarsi.

Inoltre, è dimostrato che il suono può influenzare la percezione del gusto (Cap 4). Le ricerche suggeriscono che livelli di rumore più bassi consentono ai clienti di concentrarsi maggiormente sulle sottigliezze del gusto e della consistenza, migliorando l'esperienza gastronomica complessiva. Al contrario, ambienti rumorosi e caotici possono offuscare queste percezioni, portando a un'esperienza meno piacevole.

Trovare il giusto equilibrio tra garantire il comfort acustico e creare un'atmosfera vivace è una delle sfide principali dell'acustica dei ristoranti. Molti locali moderni privilegiano spazi aperti, superfici dure e soffitti alti che, pur essendo esteticamente gradevoli, tendono ad amplificare il suono e ad esacerbare i livelli di rumore. Queste scelte progettuali possono determinare un ambiente acusticamente svantaggioso, soprattutto se combinato con un'alta densità di clienti.

Per mitigare questi problemi, i designer devono incorporare trattamenti acustici che non compromettano l'estetica dello spazio. Soluzioni come pannelli radianti a soffitto, pareti rivestite in tessuto e arredamento fonoassorbente possono contribuire a creare uno spazio in grado di bilanciare l'atmosfera con la chiarezza acustica.~\cite{wiki:sound-restaurants}

\subsection{Il concetto di Bolle Sonore}
\noindent

Il concetto di "bolle sonore ” si riferisce alla creazione di microambienti acustici personalizzati all'interno di uno spazio condiviso più ampio. In pratica, si tratta di progettare una configurazione audio in cui zone o aree diverse, come i singoli tavoli di un ristorante, possono avere paesaggi sonori isolati o personalizzati senza interferenze dai flussi audio circostanti. Le bolle sonore mirano a fornire un'esperienza uditiva mirata e coinvolgente per ogni ascoltatore o gruppo, migliorando la privacy e riducendo gli effetti negativi del rumore di fondo.

La creazione di bolle sonore richiede un preciso isolamento acustico e il controllo della distribuzione del suono. Questo può essere ottenuto attraverso una combinazione di barriere fisiche, altoparlanti direzionali e tecnologie di cancellazione del suono. I diffusori direzionali, ad esempio, concentrano i fasci audio in direzioni specifiche, impedendo al suono di diffondersi in altre aree. Allo stesso modo, i pannelli acustici o i divisori possono aiutare a bloccare le onde sonore dal passaggio oltre le zone designate.

In molti casi, la tecnologia utilizzata per creare bolle sonore coinvolge sistemi audio multi-room. Questi sistemi consentono di trasmettere flussi audio diversi alle varie parti di uno spazio, permettendo a ogni zona di avere un ambiente sonoro indipendente. Il sistema garantisce che ogni zona riceva il contenuto audio desiderato senza interferenze dalle zone adiacenti.

Snapcast, ad esempio, è un software molto diffuso che consente lo streaming audio sincronizzato su più dispositivi. Garantisce che tutti i client audio riproducano lo stesso flusso senza ritardi, rendendolo ideale per gli ambienti in cui la tempistica e il coordinamento sono fondamentali. In combinazione con Mopidy, che funge da server audio, Snapcast può trasmettere flussi audio distinti ad aree separate, creando bolle sonore controllate.

Oltre alle soluzioni basate sul software, l'hardware, come gli altoparlanti direzionali, può focalizzare il suono in direzioni specifiche, assicurando che solo il pubblico previsto lo senta chiaramente. Questi diffusori possono essere posizionati sopra o intorno a un'area specifica, ad esempio un tavolo da pranzo, per proiettare il suono direttamente nell'area di destinazione, riducendo al minimo la diffusione nelle zone adiacenti. Questa tecnologia è particolarmente utile negli ambienti open space, dove la gestione della dispersione sonora tra le zone è fondamentale.

Nelle applicazioni pratiche, le bolle sonore vengono utilizzate per creare ambienti audio personalizzati in spazi pubblici condivisi. Ad esempio, in un ristorante, ogni tavolo può avere un'esperienza sonora personalizzata, che si tratti di musica, conversazioni o altri contenuti audio. In questo modo i clienti possono godere di un'esperienza sonora più intima e controllata, senza la distrazione di conversazioni vicine o rumori di fondo. 

Oltre che nei ristoranti, le bolle sonore trovano applicazione in vari spazi commerciali e pubblici, come musei, uffici e biblioteche, dove il controllo dell'ambiente uditivo è essenziale per migliorare l'esperienza degli utenti. Questi ambienti traggono vantaggio in quanto aiutano a bilanciare privacy, concentrazione e atmosfera senza ricorrere a un isolamento acustico completo.~\cite{cit-multiaudio}

\section{Reti neurali artificiali per la previsione del gradimento alimentare}

\subsection{Fondamenti e applicazioni delle reti neurali artificiali}
\noindent


Le reti neurali artificiali (ANN) sono modelli computazionali ispirati al funzionamento del cervello umano, composti da unità di elaborazione interconnesse chiamate neuroni artificiali. Nel contesto della previsione del gradimento alimentare, le ANN offrono un potente strumento per modellare relazioni complesse e non lineari tra variabili di input (come caratteristiche acustiche e non acustiche) e output (gradimento del cibo).

L'architettura base di una ANN comprende uno strato di input, uno o più strati nascosti e uno strato di output. Ogni connessione tra i neuroni è associata a un peso $w_{ij}$, che viene ottimizzato durante la fase di addestramento della rete. La funzione di attivazione $\phi()$ introduce non linearità nel modello, permettendo alla rete di apprendere relazioni complesse. L'output di un neurone può essere espresso come:

\[O^{(l)}_i = \phi(u^{(l)}_i) = \phi(\sum_{j=1}^{n_l} O^{(l-1)}_j w^{(l)}_{j,i} + w^{(l)}_{0,i}), 1 \leq l \leq L\]

dove $O^{(l)}_i$ è l'output del neurone i nello strato l, $w^{(l)}_{j,i}$ è il peso della connessione tra il neurone j dello strato precedente e il neurone i dello strato corrente, e $w^{(l)}_{0,i}$ è il bias del neurone.~\cite{cit-FoodResearch}

L'applicazione delle ANN nel campo della percezione alimentare ha guadagnato notevole attenzione, grazie alla loro capacità di modellare relazioni complesse tra stimoli sensoriali e risposte percettive. Nel contesto specifico del gradimento alimentare in presenza di rumore di fondo, le ANN offrono un approccio promettente per catturare le interazioni multimodali tra stimoli acustici e gustativi.

\subsection{Modello \gls{ann} ottimizzato con \gls{hho}}
\noindent

Un modello \gls{ann} avanzato è stato proposto per prevedere il gradimento del cibo in presenza di diversi tipi e livelli di rumore di fondo. Questo modello, ottimizzato mediante l'algoritmo Harris Hawks Optimizer (\gls{hho}), ha dimostrato prestazioni superiori rispetto ai modelli \gls{ann} tradizionali e ai modelli statistici misti.

L'architettura del modello \gls{ann}-\gls{hho} comprende uno strato di input con cinque neuroni, corrispondenti a età, genere, sensibilità al rumore, tipo di rumore e livello di rumore. Due strati nascosti, ciascuno con dieci neuroni, elaborano queste informazioni. Lo strato di output produce la previsione del gradimento relativo del cibo. Questa architettura è illustrata nella Figura 1 del documento di riferimento.

L'algoritmo \gls{hho}, ispirato al comportamento cooperativo dei falchi di Harris durante la caccia, è utilizzato per ottimizzare i pesi e i bias della rete neurale. Il processo di ottimizzazione può essere descritto dalle seguenti equazioni:

\[X(t+1) = \begin{cases} 
   X_r(t) - r_1|X_r(t) - 2r_2X(t)| & q \geq 0.5 \\
   X_b(t) - X_m(t) - r_3(lb + r_4(ub - lb)) & \text{altrimenti}
\end{cases}\]

\[E = 2E_0(1 - \frac{t}{T})\]

Dove $X(t)$ rappresenta la posizione della soluzione al tempo $t$, $X_r$, $X_m$, e $X_b$ sono soluzioni random, media e migliore rispettivamente, $r_1$, $r_2$, $r_3$, $r_4$ sono numeri casuali, $q$ è una probabilità di cambio, $E$ è l'energia della preda, $E_0$ è il valore iniziale dell'energia, $t$ è l'iterazione corrente e $T$ è il numero massimo di iterazioni.

L'ottimizzazione \gls{hho} ha dimostrato di migliorare significativamente le prestazioni del modello \gls{ann} rispetto ad altri metodi di ottimizzazione, inclusa l'ottimizzazione a sciame di particelle (PSO). Questa superiorità è attribuibile alla capacità dell'\gls{hho} di bilanciare efficacemente l'esplorazione globale e lo sfruttamento locale dello spazio delle soluzioni, consentendo al modello di convergere verso soluzioni ottimali più rapidamente e con maggiore precisione.

\subsection{Implementazione e risultati del modello \gls{ann}-\gls{hho}}
\noindent

Il modello \gls{ann}-\gls{hho} è stato implementato utilizzando un dataset di 135 risposte, ottenute da 15 partecipanti che hanno valutato il gradimento del cibo in presenza di tre tipi di rumore a tre livelli diversi. I tipi di rumore considerati includevano musica rilassante, rumore di ristorante e rumore del traffico stradale, con livelli di 30, 40 e 50 dBA.

La performance del modello è stata valutata utilizzando diverse metriche statistiche. I risultati mostrano che il modello \gls{ann}-\gls{hho} ha raggiunto prestazioni superiori ($R^2 = 0.70$, RMSE $= 0.8$) rispetto ai modelli \gls{ann} tradizionali ($R^2 = 0.61$, RMSE $= 1.1$) e ai modelli statistici misti ($R^2 = 0.42$, RMSE $= 1.8$). Questo significativo miglioramento nelle prestazioni sottolinea l'efficacia dell'approccio di ottimizzazione \gls{hho} nel contesto della previsione del gradimento alimentare. La Figura 4 del documento di riferimento illustra graficamente queste prestazioni.

L'analisi dei risultati ha rivelato che i livelli di rumore tra 30 e 35 dBA producono il massimo gradimento relativo del cibo per diversi tipi di rumore. In particolare, un livello di 30 dBA ha dato il massimo gradimento relativo per la musica e il rumore di ristorante, mentre 35 dBA ha prodotto il massimo gradimento per il rumore del traffico stradale.

Inoltre, l'analisi delle caratteristiche utilizzando l'algoritmo ``RReliefF'' ha mostrato che i fattori acustici, come il tipo e il livello di rumore, hanno un peso maggiore rispetto ai fattori non acustici nella previsione del gradimento del cibo. La Figura 6 del documento di riferimento illustra l'importanza relativa di questi fattori.

È interessante notare che il modello ha previsto che la musica potrebbe ancora essere accettabile fino a un livello di 47 dBA, mentre il rumore del traffico stradale e il rumore del ristorante hanno prodotto valutazioni di gradimento del cibo significativamente negative a tutti i livelli studiati.

Questi risultati forniscono preziose indicazioni per la gestione del rumore nelle aree di ristorazione e potrebbero contribuire a strategie più mirate per la valutazione e la mitigazione del rumore in questi ambienti.